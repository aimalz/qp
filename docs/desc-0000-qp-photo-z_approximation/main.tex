\RequirePackage{docswitch}
\setjournal{\flag}

\documentclass[\docopts]{\docclass}


\usepackage{lsstdesc_macros}

\usepackage{xspace}
\usepackage{graphicx}
\graphicspath{{./}{./figures/}{.logos}}
\bibliographystyle{apj}


\newcommand{\textul}{\underline}
\newcommand{\qp}{\texttt{qp}\xspace}
\newcommand{\pz}{photo-$z$ PDF\xspace}
\newcommand{\Pz}{Photo-$z$ PDF\xspace}


\begin{document}

\title{ Approximating photo-z PDFs for large surveys }


\begin{abstract}

Upcoming and ongoing galaxy surveys will produce redshift probability
distribution functions (PDFs) in addition to traditional photometric redshift
(photo-$z$) point estimates.  However, the storage of \pz s may present a
challenge with increasingly large catalogs, as we face a trade-off between the
accuracy of subsequent science measurements and the storage cost.  This paper
presents \qp, a Python package facilitating manipulation of approximations of
1-dimensional PDFs, as suitable for \pz s.  We use \qp\ to investigate the
performance of three simple PDF storage formats on two realistic mock datasets, representative of
upcoming surveys with different data qualities, as a function of the number of
stored parameters per \pz, using metrics of both individual \pz s and an
estimator of the overall redshift distribution function.

\end{abstract}

\dockeys{methods: data analysis, catalogs, surveys}

\maketitlepost





\section{Introduction}
\label{sec:intro}


Ongoing and upcoming wide field imaging surveys such as that planned with the Large Synoptic
Survey Telescope (LSST) will observe billions of galaxies; studies of cosmology
and galaxy evolution with these data will rely on the method of photometric redshift
(photo-$z$) estimation.  Photo-$z$s are subject to a number of systematic
errors, some caused by the estimation procedure and others intrinsic to the
data itself.  The photo-$z$ community has come to favor
methods that provide a redshift probability distribution function (PDF) $p(z)$ for each galaxy in the survey,
that includes information about the potential for such systematic errors.
% These \Pz s are interim posterior distributions, as they
% are estimates of the probability of a galaxy's redshift conditioned on its
% photometric data and any assumptions made by the method producing it, though
% they are commonly written simply as $p(z)$.

Given the tremendous size of the surveys in question, storage of these
probability distributions involves making difficult decisions.  Each survey
seeks to create a catalog of \pz s balancing accuracy against storage
cost.  For example, the \pz\ catalog that LSST will release will be limited to
200 floating point numbers per galaxy, with plans to store \pz s derived by
multiple methods \citep{juric_data_2017}.
The problem of \pz approximation for large surveys was first addressed in \citet{carrasco_kind_sparse_2014} in the
context of a single galaxy survey, a limited set of \pz approximation schemes, and
metrics restricted to the mean accuracy of the individual \pz s. However, we expect the choice of \pz approximation, and the number of
stored parameters associated with it, will
depend on the science case and its requirements on \pz accuracy: different science cases will need different accuracy metrics.  In this paper we address the question of \textit{how} these choices should be
made in general, by providing a publicly available \qp\ Python package  to enable each survey to optimize their \pz approximation via mathematically motivated and science-driven metrics. We
demonstrate this approach on two sets of realistic mock data.

In Section~\ref{sec:methods}, we outline how \qp\ can be used to optimize the
choice of \pz approximation.  In Section~\ref{sec:data}, we
describe the mock datasets on which we perform such an analysis.  We present
the results of this procedure in Section~\ref{sec:results} and make
recommendations for the use of \qp\ by the photo-$z$ community in Section~\ref{sec:conclusions}.








\section{Methods}
\label{sec:methods}

We have developed the \qp\ Python package to facilitate the representation and approximation of \pz s.
A \texttt{qp.PDF} object can carry a number of different approximations, each with its own parametrization, and can handle the  conversions between different parametrizations.  The currently supported parametrizations are described in
Section~\ref{sec:approx}. The \qp\ package also provides a few built-in metrics to quantify the
accuracy of a representation of a \pz\ relative to a given
parametrization that has been designated as ``true.''  The currently implemented metrics are
described in Section~\ref{sec:metric}. Large-scale tests can be conducted using the
\texttt{qp.Ensemble} class, that provides a wrapper for collections of \texttt{qp.PDF} objects.

\subsection{Approximation Methods}
\label{sec:approx}

First, we establish a vocabulary for the definitions of approximation methods.
Each \textit{parametrization} of a \pz\ is defined in terms of the
\textit{format} function $\mathcal{F}$, \textit{metaparameters}
$\vec{C}$, and \textit{parameters} $\vec{c}$.  This parametrization
then corresponds to a \textit{representation}
\begin{align}
  \label{eq:definition}
  \hat{p}_{\mathcal{F}, \vec{C}, \vec{c}}(z) &\equiv \mathcal{F}_{\vec{C}}(z;
\vec{c})
\end{align}
of the \pz, denoted as $\hat{p}(z)$ for brevity.  We often employ interpolation
schemes with a generic interpolator functions $F_{\vec{C}'}(z; z', p')$ that
comes with its own metaparameters $\vec{C}'$.
% The choice of interpolator may
% be made on a survey-by-survey basis.
\qp\ supports all interpolation options
available to the \texttt{scipy.interpolate.interp1d} function,  but we
have chosen a default interpolation scheme for each format to maximize its
performance.

\qp\ supports conversion of \pz\ approximations between five formats: step functions,
samples, quantiles, evaluations, and mixture model components.  These formats
may be associated with any number $N_{f}$ of stored parameters $c_{i}$ per \pz.
% which are presumed to be floating point numbers unless otherwise specified.
Meanwhile, the metaparameters $C_{i}$ are the set of numbers necessary to convert the stored
\pz\ parameters $\vec{c}$ into a probability distribution function over
redshift.  In this work we consider special cases of three of these formats as candidates
for large survey \pz\ catalog storage: regular binning (Section~\ref{sec:bins}),
random samples (Section~\ref{sec:samples}), and regular quantiles (Section~\ref{sec:quantiles}), while the other two, evaluations on a grid and mixture
model components, are used solely for internal manipulations within \qp.

We have not yet included the \texttt{SparsePz} sparse basis representation of
\citet{carrasco_kind_sparse_2014}, which uses a mixture model of $N_{f}$
members of a library of $\sim10^{4}$ functions and has impressive compression
properties.
% A mixture model is only appropriate when the true \pz\ is
% in fact a mixture of the functions in the model, which we cannot in general
% check with real data.
However, decomposition with \texttt{SparsePZ} does not enforce that the stored parametrization be a probability
distribution in the mathematical sense of integrating to unity and, especially,
always being positive semidefinite.
While normalizing a positive semidefinite
function to integrate to unity is always possible (if the endpoints of
integration are specified), one can motivate multiple schemes for enforcing
nonnegativity that result in different reconstructions $\hat{p}(z)$.
% To
% discourage misuse of \qp, we do not permit mixtures of functions that can take
% negative values at this time.
We postpone the inclusion of the sparse basis
representation in \qp\ and investigation of these non-negativity schemes to future work.

The various simple \qp\ formats are illustrated in Figure~\ref{fig:qp} on an example multimodal \pz.
\begin{figure}
  \includegraphics[width=0.9\columnwidth]{figures/demo_pz.png}
  \caption{\qp\ approximation of a continuous 1D PDF (solid black line) using
the step function (orange dotted line), random samples (green dash-dotted line),
and quantile formats (purple dashed line), each with the same number of stored parameters ($N_{f}=20$).
  \label{fig:qp}}
\end{figure}

For each format, we address the following questions:
\begin{itemize}
  \item When/where has this format appeared in the literature as a published
catalog format, native \pz\ code output format, and science application input
format?
  \item What exactly is stored under this format, per galaxy (the parameters)
and per catalog (the metaparameters)?
  \item Beyond fidelity to the original \pz, what are the a priori strengths
and weaknesses of this format?
\end{itemize}


\subsubsection{Regular Binning}
\label{sec:bins}

By far the most popular format for approximating and storing \pz s in the literature is that of a piecewise constant step
function, also called a histogram binning.  It is the only format that has been
used for public release of \pz\ catalogs \citep{tanaka_photometric_2017,
sheldon_photometric_2012}; it is unclear whether this is a consequence or cause
of the fact that it is the most common format for using \pz s in cosmological
inference, as tomographic binning is a universal step between the \pz\ catalog
and calculation of any two-point correlation function.

The metaparameters of the binned parametrization are the ordered list of
redshifts $\vec{C} = (z_{1}, z_{2}, \dots, z_{N_{f}}, z_{N_{f}+1})$ serving as
bin endpoints shared by all galaxies in the catalog, each adjacent pair of
which is associated with a parameter $c_{i}=\int_{C_{i}}^{C_{i+1}}\ p(z)dz$.
The \qp\ histogram format assumes $p(z)=0$ when $z<C_{1}$ or $z>C_{N_{f}+1}$ leading
to the normalization condition $\sum_{i} c_{i}(C_{i+1}-C_{i})) = 1$.\footnote{Note that
this is not generally equivalent to the erroneous normalization condition
$\sum_{i} c_{i} = 1$ commonly enforced in public catalogs.}  The histogram
format function $\mathcal{F}^{h}$ is the sum of a set of $N_{f}$ step
functions, making the reconstructed estimator of the \pz
\begin{align}
  \label{eq:binned}
  \hat{p}^{h}(z) &= \sum_{i}^{N_{f}}\ c_{i}
\left\{\begin{tabular}{cc}$1$&$C_{i}<z<C_{i+1}$\\
0&$z < C_{i}$\ or\ $z > C_{i+1}$\end{tabular}\right\},
\end{align}
where the step functions may be considered their own interpolators.  Here we
only consider a regular binning, with $C_{i+1}=C_{i}+\delta_{f}$ for a constant
$\Delta$, as this is the only type of binning that has been used in the
literature.

The regular histogram format may be considered wasteful in terms of data storage. A \pz\ with a very compact (broad) probability distribution may have many
parameters taking the same value $c_{i}\approx0$
($c_{i}\approx(C_{N_{f}+1}-C_{1})\delta_{f}^{-1}$) that are redundant in
storage.
% It also requires the researcher to choose the minimum and maximum
% possible redshifts $C_{1}$ and $C_{N_{f}+1}$ of the galaxy sample, which are
% unknown quantities, so it would be preferable to leave them unconstrained when
% setting up the catalog.

\subsubsection{Random Samples}
\label{sec:samples}

Samples are the native output format of many machine learning algorithms
dependent on random choices, such as random forests.
Such approaches typically produce large numbers of samples, far more than can
realistically be stored by any survey, so a subsample is commonly stored.

The parameters of the samples format are the set of $N_{f}$ samples
$\vec{c}=(z_{1}, z_{2}, \dots, z_{N_{f}-1}, z_{N_{f}})$, where $C=N_{f}$ is an
implicit metaparameter.  Though it is possible to construct a catalog where $C$
is not uniform over the catalog but is instead optimized somehow for each galaxy, we leave its investigation to future work, as
it has not yet appeared in the literature.  The format function
$\mathcal{F}^{s}$ is needed to turn the samples into a representation of the \pz; in the tests
presented here, we take $\mathcal{F}^{s}$ to be the Gaussian kernel density estimate (KDE) of
\texttt{scipy.stats.gaussian\_kde} with the smoothing bandwidth $C'$ as a
metaparameter.  The samples representation is then
\begin{align}
  \label{eq:sampled}
  \hat{p}^{s}(z) &= \mathcal{F}^{s}_{C'}(z; \vec{c}).
\end{align}

Though samples are an obvious choice for \pz s with narrow features with high
amplitude, we expect that using only a small number of samples from a broad \pz\ will increase the variance of any ensemble metrics
(as the sampling involves additional shot noise).  The researcher must also choose an interpolation method to
reconstruct a \pz\ from samples.


\subsubsection{Regular Quantiles}
\label{sec:quantiles}

One parametrization that has not previously been investigated is that of
quantiles, which are defined in terms of the cumulative distribution function
(CDF).  Under the quantile format, a \pz\ catalog shares $N_{f}$ ordered CDFs
$\vec{C}=(q_{1}, q_{2}, \dots, q_{N_{f}-1}, q_{N_{f}})$.  Each galaxy's catalog
entry is the vector of redshifts $\vec{c}=(z_{1}, z_{2}, \dots, z_{N_{f}-1},
z_{N_{f}})$ satisfying $CDF(c_{i})=C_{i}$, so the quantile format function
$\mathcal{F}_{q}$ is the derivative of an interpolation of the inverse CDF
$CDF^{-1}(C_{i})=c_{i}$ under the interpolation scheme $F$.  In this study, we
test regular quantiles $C_{i}\equiv i(N_{f}+1)^{-1}$ using a linear
interpolation scheme, but note that either of these choices could be replaced.  The format function is then the convolution
$\mathcal{F}^{q}=F(CDF^{-1})$, making the quantile representation
\begin{align}
  \label{eq:quantiles}
  \hat{p}_{q}(z) &= F(z; \vec{c}, \frac{\Delta\vec{C}}{\Delta\vec{c}}).
\end{align}
% In our tests, the quantiles are regular, such that $C_{i}\equiv i\bar{C}$,
% where $\bar{C}\equiv(N_{f}+1)^{-1}$, but \qp\ does not require that this be so.
%  Additionally, in the tests presented here, we use a linear interpolator by
% default for the quantile format.


We expect the quantile format to be an efficient approximation for \pz\ s, because it
allocates storage evenly in the space of probability density. In contrast, the
histogram format stores data evenly spaced in redshift, and the samples format
stores data randomly in probability density.  As with the samples
representation, an interpolation function must be chosen for reconstructing the
\pz\ from the stored parameters.  Depending on the native \pz\ output format,
converting to the quantile format may require $N_{f}$ numerical optimizations to find the quantile points. We accelerate these optimizations by first finding rough approximate quantiles by simple approximation of the CDF on a grid.





\subsection{Comparison Metrics}
\label{sec:metric}

In this work, our aim is to probe how
closely \pz s reconstructed from limited sets of stored parameters approximate the
``true'' (or exact) PDF. This can be done without reference to a galaxy's true redshift.
(For a demonstration
of how one might approach the different problem of evaluating the accuracy of a
\pz\ relative to a true redshift, see Schmidt et al.\ in preparation.)  The loss of information incurred when using an approximate PDF
$\hat{P}(z)$ instead of the underlying true PDF $P(z)$ is given by
the Kullback-Leibler divergence (KLD), which is defined as
\begin{align}
  \label{eq:kld}
  KLD[P(z) || \hat{P}(z)] &= \int_{-\infty}^{\infty}\ P(z)\
\log\left[\frac{P(z)}{\hat{P}(z)}\right]\ dz\\
  &\approx \delta_{ff}\sum_{z=z_{1}}^{z_{N_{ff}}}\ P(z)\
\log\left[\frac{P(z)}{\hat{P}(z)}\right],
\end{align}
where $\log$ is the natural logarithm throughout this paper.  Because there is
in general no closed-form expression for the KLD, we calculate the KLDs in this
paper using evaluations of the PDF under each format on a very fine, regular
grid $(z_{1}, z_{2}, \dots, z_{N_{ff}-1}, z_{N_{ff}})$ with resolution
$\delta_{ff}\ll\delta_{f}$.  We review the properties of the KLD and provide some
intuition for it in Appendix~\ref{sec:kld}.

\subsubsection{Individual \pz s}
\label{sec:individual_metric}

Some science applications rely on the recovery of individual galaxy \pz s that,
for example, may be used as the basis for targeting spectroscopic follow up.  For this purpose, we calculate the KLD
for each individual \pz\ in our catalogs, and then characterize the distribution of KLD values across the sample
% $\hat{p}(KLD)$
by its first, second, and third moments, (the mean,
variance, and kurtosis).  We use these aggregate statistics to observe how the
approximate individual galaxy \pz s for some dataset vary with the choice of parametrization.


\subsubsection{Stacked $\hat{n}(z)$ estimator}
\label{sec:stacked_metric}

In cosmology, \Pz s have thus far been used almost exclusively to estimate the redshift
distribution function $n(z)$ necessary for calculating the correlation
functions used by many cosmological probes.  The most common way to estimate
the redshift distribution function for a sample of $N_{g}$ galaxies is to sum the \pz s according to
\begin{align}
  \label{eq:nz}
  \hat{n}(z) &\equiv \frac{1}{N_{g}}\ \sum_{k=1}^{N_{g}}\ \hat{p}_{k}(z).
\end{align}
Here, the
estimator is normalized, such that it, too, is a probability distribution.
%  (though
% in the literature it is often subject to the fallacy conflating a sum and an
% integral, first mentioned in Section~\ref{sec:bins}).
While we do not recommend this
approach to estimating the redshift distribution,
% (see Malz and Hogg, et al. (in
% prep.) for a mathematically principled alternative),
we use it here on the grounds that i) any metric calculated on a more principled estimator of the
redshift distribution function will have similar behavior with respect to the
parametrization of the \pz\ catalog, and ii) it is easy to compute.  Our
primary metric is therefore the KLD between the stacked estimator of a catalog of
evaluations of reconstructed \pz s, to the stacked estimator of a catalog of
evaluations of the true \pz s.


\section{Photo-z Test Data}
\label{sec:data}

With the expectation that the optimal parametrization for approximating \pz s may differ according to the properties of the original photometric data, we demonstrate a procedure for vetting \pz\
parametrizations on a pair of mock datasets, each intended to be realistic
projections of the anticipated LSST \pz s.  All \pz s were inferred using the
publicly available Bayesian Photometric Redshift (BPZ) code
\citep{benitez_bayesian_2000}, which employs spectral energy distribution (SED)
fitting to a template library.

The datasets differ in the sets of SEDs used in generating the mock photometry and then inferring the $p(z)$, and so can be thought of as coming from two different \pz estimation methods.
The choice of \pz\ estimation method, however,
is not relevant to this study; so long as the mock \pz s are
\textit{realistically complex}, meaning they take shapes similar to those we expect
to see in \pz s from real datasets with similar photometric properties, it does
not matter whether the \pz s produced by BPZ are accurate redshift posteriors.
We seek only to optimize the fidelity of the stored \pz\ relative to the \pz\
output by a representative \pz\ fitting code.  \citep[See][and Schmidt et al. in preparation for other work
comparing the accuracy of \pz s produced by different methods.]{tanaka_photometric_2017}.
% As BPZ is a
% widely used and well established method, we assume that the \pz s produced by
% it are of realistic complexity.
% The default storage format of a BPZ \pz is a $N_{ff}>200$
% gridded parametrization. This resolution exceeds the available storage for any
% planned survey.
Because we believe that each galaxy has an underlying redshift
interim posterior probability distribution that is a continuous function, to
which the output of BPZ is itself a high-resolution approximation, we fit the
gridded BPZ \pz\ with a Gaussian mixture model that we then designate as the ``true'' \pz for our accuracy tests.


\subsection{Higher quality mock data}
\label{sec:graham}

Our first dataset is a $N_{g}\approx30,000$ object subset of the simulated galaxy
catalog used for LSST photometric redshift experiments by Graham et al. (in
preparation). This is based on the Millennium simulation
\citep{springel_simulations_2005}, and in particular the LC DEEP Gonzalez2014a
catalog based on the galaxy formation models of \cite{gonzalez-perez_how_2014},
and was created using the lightcone construction techniques described by
\cite{merson_lightcone_2013}.  We limit the sample to galaxies with a catalog
$i$-band magnitude of $i<25$ and true redshifts $z<3.5$. As in Graham et al.
(in preparation), we simulate observed apparent magnitudes from the true catalog
magnitudes by adding a normal random scatter with a standard deviation equal to
the predicted magnitude error for each galaxy (from Section 3.2.1. of
\citealt{ivezic_lsst:_2008}, using the software of
\citealt{connolly_end--end_2014}, assuming a mean airmass of 1.2 and a 10-year
accumulation of 56, 80, 184, 184, 160, and 160 visits in filters $ugrizy$,
respectively).  We also ignore any magnitudes fainter than the predicted
10-year limiting magnitudes in each filter, $u<26.1$, $g<27.4$, $r<27.5$,
$z<26.1$, and $y<24.9$, as a realistic simulation of non-detections.

The \pz estimates for this simulated catalog use the CFHTLS set of spectra
\citep{ilbert_accurate_2006} as templates, and the default parameter settings for BPZ, except
that we impose a maximum photometric redshift of 3.5 and allow BPZ to use the
$i$-band as a magnitude prior during the photo-$z$ fit. The \pz s from BPZ are
in the form of $N_{ff} = 351$ evaluations of the probability density on a
regular grid of redshifts $0.01 < z < 3.51$, a subsample of which are plotted
in Figure~\ref{fig:graham_pzs}.

As this Figure shows, the \pz s tend to be unimodal and sharply peaked, as if coming from ``higher quality'' photometric data.
We produce true \pz s for the analysis by
fitting a three-component Gaussian mixture model to each \pz\ in the catalog. We then calculated the three different approximations to each \pz, and evaluated their accuracy using the metrics described above.

\begin{figure}
  \includegraphics[width=0.9\columnwidth]{figures/graham_pzs.png}
  \caption{Example \pz s from the mock LSST data of Graham et al. The mock photometry yields largely narrow, unimodal \pz s.
  \label{fig:graham_pzs}}
\end{figure}


\subsection{Lower-quality mock data}
\label{sec:LSST}

Our second dataset is an independent simulation of the expected LSST galaxy sample. Here, we use the
Buzzard-highres-v1.0 mock galaxy catalog of deRose, et al. (in preparation) of
galaxies with SEDs drawn from an empirical library of $\sim500,000$ SEDs from
the Sloan Digital Sky Survey (SDSS).
Given an SED, redshift, and absolute $r$-band magnitude for each galaxy, we
compute the expected apparent magnitudes and magnitude errors in the six
broadband LSST filters ($ugrizy$), assuming the full 10-year depth of the survey
using the simple model of \citet{ivezic_lsst:_2008}.  The catalog contains
$N_{g}\approx100,000$ galaxies to a depth of $i<26.9$, 1.5 magnitudes deeper
than the expected ``LSST Gold sample'' of galaxies that will have $S/N\gtrsim30$
in multiple bands.

In implementing BPZ, we created a custom Bayesian prior using a subset of the
Buzzard-highres-v1.0 catalog and a spanning template set via a simple k-means
clustering algorithm based on $100$ of the SDSS SEDs used in creating the
Buzzard catalog.  BPZ produces \pz s in the format of probability density
evaluations on a regular grid of $N_{ff}=211$ redshifts $0.005\leq z\leq2.105$,
a subsample of which are plotted in Figure~\ref{fig:lsst_pzs}.  Even with six
filters spanning the optical, there are known degeneracies (e.g.~the
Lyman/Balmer break degeneracy) that lead us to expect the presence of
multi-modal \pz s.  We produce true \pz s for the analysis by fitting a
five-component Gaussian mixture model to each gridded \pz\ in the catalog. We then calculated the three different approximations to each \pz, and evaluated their accuracy using the metrics described above.

\begin{figure}
  \includegraphics[width=0.9\columnwidth]{figures/lsst_pzs.png}
  \caption{Example \pz s from the mock LSST data of deRose et al. This sample contains a higher proportion of broader,
multimodal \pz s, as if from lower quality data.
  \label{fig:lsst_pzs}}
\end{figure}


\section{Results \& Discussion}
\label{sec:results}

We calculate the metrics of Section~\ref{sec:metrics} on 10 random
instantiations of catalogs of $N_{g}=100$ galaxies drawn randomly from each of the datasets discussed in
Section~\ref{sec:data}, and each of $N_{f}=3,\ 10,\ 30,\ 100$ stored parameters.
We then illustrate how our results
could be used to choose an appropriate parametrization, given constraints on
$KLD_{lim}$, $p(KLD)$, or $N_{f}$.

\subsection{Stacked $\hat{n}(z)$ estimator}
\label{sec:stacked_results}

Figure~\ref{fig:stacked} shows an example of $\hat{n}(z)$ estimated from \pz s reconstructed
from just $N_{f}=10$ parameters, under each of our three approximation formats. As expected, the stacked histogram (on the same grid as the approximated $p(z)$'s) is quite coarse. The samples and quantiles can be interpolated such that the stacked $n(z)$ estimator is close to the true stacked $n(z)$ estimator.

\begin{figure}
  \includegraphics[width=0.9\columnwidth]{figures/graham_stacked.png}
  \caption{An example of the stacked estimator of the redshift
distribution, for a subsample of $N_{g}=100$ galaxies drawn from the higher-quality dataset and
with just $N_{f}=10$ parameters used for each $p(z)$. The
most striking characteristic of $\hat{n}(z)$ with a relatively small number of
parameters on a small number of galaxies is the coarseness of the histogram
format (orange dotted line) relative to the quantile format (purple dashed
line) and samples format (green dash-dotted line), both of which are fairly
close to $\hat{n}(z)$ derived from evaluating the true \pz s (thick gray line).
  \label{fig:stacked}}
\end{figure}

The $\hat{n}(z)$ KLD values for each parametrization on both mock datasets are
collected and plotted in Figure~\ref{fig:kld}, with error regions estimated from the variance between the 10 instantiations.  We can immediately see that [lower KLDs
are achievable with the higher quality dataset] and [the discrepancy between
formats is greater for the higher quality dataset], indicating that [narrower,
unimodal distributions are more sensitive to the parametrization]\dots

\begin{figure}
  \includegraphics[width=0.9\columnwidth]{figures/lsst_kld.png}\\
  \includegraphics[width=0.9\columnwidth]{figures/graham_kld.png}
  \caption{KLD between approximate $n(z)$ estimator and true $n(z)$ estimator as a function of number of stored
parameters, for three different approximation schemes: quantiles (purple dashed line), samples (green dash-dotted
line), and histogram (orange dotted line).  Top panel: The histogram
format is not well-suited to the lower-quality dataset of Section~\ref{sec:LSST},
while the quantiles and samples formats perform comparably well.  Bottom panel: The
higher-quality dataset of Section~\ref{sec:graham} favors the quantiles format and
strongly disfavors the histogram format, with samples being slightly less
favorable than quantiles.
  \label{fig:kld}}
\end{figure}

We interpret Figure~\ref{fig:kld} in the context of constraints on storage
allocation imposed by the survey, and constraints on the acceptable degree of
information loss imposed by the science requirements.  The former corresponds
to a vertical line at $N_{f, lim}$ in Figure~\ref{fig:kld}; the best format would
be the one that achieves the lowest KLD at $N_{f, lim}$.  The latter
corresponds to a horizontal line at $KLD_{lim}$ in Figure~\ref{fig:kld}; the best
parametrization would be the one that achieves $KLD_{lim}$ at the smallest
value of $N_{f}$.

If there is some flexibility in the acceptable degree of information loss on
$\hat{n}(z)$ and/or the allocation of storage for \pz s, as is the case for
LSST, it may be best to examine the asymptotic behavior of the KLD as a
function of $N_{f}$ for each format considered.
% , in order to make a choice based on a
% marginal difference.
For example, if the KLD can be significantly reduced with
a slightly larger $N_{f}$, it may be possible to request additional storage
capacity for the survey's \pz s.  We note that for these datasets, the KLD
achieves its asymptote at different values of $N_{f}$ for each format, with the
histogram format needing significantly larger $N_{f}$ than the other formats.

\subsection{Individual \pz s}
\label{sec:individual_results}

We also compare our three parametrizations on the basis of the distributions of the KLD
of Section~\ref{sec:metric} calculated for all \pz s in each
dataset.  An example of the individual \pz\ KLD distribution is shown in Figure~\ref{fig:individual} for the low-quality dataset with $N_{f}=10$.

\begin{figure}
  \includegraphics[width=0.9\columnwidth]{figures/lsst_individual.png}
  \caption{The distribution of log-KLD
values for $N_{g}=100$ \pz s from the low-quality dataset with $N_{f}=10$ over
the quantiles (purple with dashed border), samples (green with dash-dotted
border), and histogram (orange with dotted border) formats.  In this case, the
quantile format has a narrow distribution with a higher KLD, whereas the
samples and histogram formats have broader distributions with generally lower
KLDs.
  \label{fig:individual}}
\end{figure}

To distill what is observed in plots like Figure~\ref{fig:individual} for both
datasets and all parametrizations, we compare the moments of the distributions
of metric values for the distribution of the KLDs of individual \pz s under
each parametrization, summarized in Figure~\ref{fig:moments}.  While it is
obvious that one would like the mean (first moment) of the KLD distribution to
be low, interpretation of higher-order moments is less clear.  In a science
application that is robust to \pz\ outliers, a parametrization with a high
variance (second moment) may be acceptable, whereas in another science
application that simply requires well-characterized errors could tolerate a
higher mean in exchange for a lower variance.  To meaningfully interpret the
KLD of individual \pz s, it will be necessary for those using \pz s in their
science to calculate the requirements on the acceptable degree of information
loss.

\begin{figure}
  \includegraphics[width=0.9\columnwidth]{lsst_moments.png}\\
  \includegraphics[width=0.9\columnwidth]{graham_moments.png}
  \caption{The mean ($\bigstar$), variance ($+$), and kurtosis ($\times$) of
the KLD distributions are plotted for each dataset over all parametrizations
(quantiles in purple, samples in green, and histogram in orange).  The moments
and their error regions have been offset about $N_{f}$ to improve readability.
Top panel: The lower-quality dataset has\dots.  Bottom panel: The higher quality
dataset has\dots.
  \label{fig:moments}}
\end{figure}







\section{Conclusions \& Future Directions}
\label{sec:conclusions}


This work develops a principled approach to choosing a storage format and
resolution for catalogs of \pz s, captured in the Python package \qp. Different approximation schemes can be applied and compared, via the Kullback-Leibler Divergence between the approximating and true $p(z)$'s. We simplemented three simple approximation schemes: a regular histogram of $N_f$ bins, a set of $N_f$ samples interpolated via a simple Gaussian KDE, and a set of $N_f$ regularly spaced quantiles with linear interpolation between them.
We tested these three methods on two realistic mock LSST \pz\ catalogs and draw the following conclusions:
\begin{itemize}
  \item The samples and quantiles approximations provide comparable accuracy ... cf histogram
  \item Flat KLD in Nf for samples and quantiles? How low can we go in Nf?
  \item The quantile parametrization is a promising
option for minimizing loss of information in \pz\ storage.
  \item
\end{itemize}

Given the constraint that LSST will be able to store 200 floating point
numbers to quantify the redshift of each galaxy and intends to include several
\pz\ codes, we can safely say that LSST can store the output of more than one
\pz\ code without any significant risk of loss of information.

We invite the community to contribute additional formats and metrics
to the publicaly available \qp Python package developed for this project.  \qp\ is a tool that can be used to optimize the choice of
stored parametrization and number of stored parameters of a catalog of \pz s
based on the accuracy needs of the use cases of the catalog.
% Further
% applications of \qp\ functionality for manipulations of \pz s will be demonstrated in the LSST-DESC PZ DC1 paper (in prep.).
We do not advocate for a one-size-fits-all solution to the problem and
emphasize that the optimal choice must depend on the requirements of the
science metric(s) and characteristics of the underlying \pz\ catalog.
% Furthermore, this procedure does not address the computational resources that
% may be necessary to perform the storage operation and to reconstruct \pz s into
% the format necessary for science calculations.


\subsection*{Appendix}
\label{sec:kld}

The Kulback-Liebler divergence measures the loss of information, measured in nats, due to using an
approximation of a distribution.  The most important feature of the KLD is its
asymmetry; it is not a distance, like the root mean square error, that is the
same from $P(z)$ to $P'(z)$ as it is from $P'(z)$ to $P(z)$.
The KLD requires that both functions $P(z)$ and $P'(z)$ be
true probability distributions (always positive semi-definite and integrating to
unity); this may need to be explicitly enforced for
some PDF approximations.  The KLD is always
positive, and a smaller value indicates better agreement between the
approximation and the truth.

To develop some intuition, consider the simple example of a Gaussian
$P(z)=\mathcal{N}(\mu_{0}, \sigma_{0}^{2})$ being approximated by a Gaussian
$P'(z)=\mathcal{N}(\mu, \sigma^{2})$, whose KLD is
\begin{align}
  \label{eq:gaussian}
  KLD &= \frac{1}{2}\left(\log\left[\frac{\sigma^{2}}{\sigma_{0}^{2}}\right] +
\frac{\sigma_{0}^{2}}{\sigma^{2}} + \frac{(\mu-\mu_{0})^{2}}{\sigma^{2}} -
1\right)
\end{align}
To get a sense of the units of information, we can calculate the KLD in some
limiting cases.  If $\sigma=\sigma_{0}$ but $\mu=\mu_{0}+1$, we obtain
$KLD=\frac{1}{2}$ in units of nats -- if the mean of the approximation is wrong
by an additive factor of $1\sigma$, half a nat of information is lost.  If
$\mu=\mu_{0}$ but $\sigma=\sqrt{2\pi}\sigma_{0}$, we find
$KLD\approx\frac{1}{2}$ in units of nats -- half a nat of information is also
lost if the variance of the approximation is off by a multiplicative factor of
$2\pi$.

We can use the KLD to identify notions of imprecision and inaccuracy.
Intuitively, precision must be related to how close $\sigma$ is to $\sigma_{0}$
and accuracy must be related to how close $\mu$ is to $\mu_{0}$.  If
$\mu\approx\mu_{0}$, we can say $KLD\sim\log[r] + \frac{1}{2}r^{-2} -
\frac{1}{2}$ where
$r^{-1}\equiv\frac{\sigma_{0}}{\sigma}$ is a measure of ``precision," whose behavior is illustrated in Figure~\ref{fig:precision}.
\begin{figure}
  \includegraphics[width=0.9\columnwidth]{figures/kld_precision.png}
  \caption{The KLD (solid black line) is proportional to the log of the inverse
precision $r$ for $\sigma>\sigma_{0}$.
% but is more complicated for $\sigma<\sigma_{0}$.
  \label{fig:precision}}
\end{figure}
We observe that an overestimated variance increases the KLD as the log of the
square root of the ratio of the estimated variance to the true variance.  When
$\sigma\approx\sigma_{0}$, $KLD\sim t^{2}$ in terms of the ``tension"
$t\equiv\frac{(\mu-\mu_{0})^{2}}{\sigma^{2}}$, whose behavior is illustrated in
Figure~\ref{fig:tension}.
\begin{figure}
  \includegraphics[width=0.9\columnwidth]{figures/kld_tension.png}
  \caption{The KLD (solid black line) is equal to the square of the tension
$t$, with a small additive offset when $r\neq1$.
  \label{fig:tension}}
\end{figure}




\subsection*{Acknowledgments}

The work of PJM was supported by the U.S. Department of Energy under
contract number DE-AC02-76SF00515.
SJS was partially supported by the National Science Foundation under grant
N56981CC.


\input{acknowledgments}

Author contributions are listed below. \\
A.I.~Malz: Initiated project, led development work. \\
P.J.~Marshall: Advised on statistics, and project design and management. \\
S.J.~Schmidt: Provided the Optical dataset. \\
M.~Graham: Provided the Optical+IR dataset. \\
J.~DeRose: Contributed to the production of the Optical dataset. \\
R.~Wechsler: Contributed to the production of the Optical dataset. \\



\bibliography{lsstdesc,main}

\end{document}
